\documentclass[UTF8]{ctexart}

\title{\Large 中国科学技术大学\\{\Large 电子技术实验III}\\{\Large 实验报告}}
\usepackage{amsmath}
\usepackage{amsfonts}
\usepackage{amssymb}
\usepackage{bm}
\usepackage{enumerate}
\usepackage{geometry}
\geometry{left=2.5cm,right=2.5cm,top=3cm,bottom=3.5cm}
\usepackage{fancyhdr}
\usepackage{lastpage}
\pagestyle{fancy}
\fancyhead[l]{ }
\fancyhead[r]{ }
\fancyhead[C]{
	\begin{tabular}{cccccc}
         & \multicolumn{4}{c}{\textbf{模拟乘法器调幅、解调与峰值检波\quad 实验报告}}                                    &            \vspace{1ex}\\
信息科学技术学院 & \multicolumn{2}{c}{PB22051030 王旭东} & \multicolumn{2}{c}{PB22051031 李毅} & 2024年11月22日
\end{tabular}
}
\fancyfoot[C]{ 第 {\thepage} 页,共 \pageref{LastPage} 页}
\setlength{\headheight}{29.83218pt}
\setlength{\abovecaptionskip}{1em}
\renewcommand{\headrulewidth}{1pt}
\usepackage{graphicx,tikz}
\usepackage{geometry}
\usepackage[hidelinks]{hyperref}
\usepackage{multicol}
\usepackage{multirow}
\usepackage{ragged2e}
\usepackage[square,comma,numbers,super]{natbib}
\bibliographystyle{unsrt}
\usepackage{siunitx}
\usepackage{subcaption}
\usepackage{wrapfig}
\usepackage{xcolor}
\usepackage{cite}
\usepackage{booktabs}
\usepackage{diagbox}
\usepackage{listings}
\usepackage{makecell}
\usepackage[final]{pdfpages}
\usepackage[T1]{fontenc}
\usepackage{float}
\makeatletter
\newcommand\dlmu[2][4cm]{\hskip1pt\underline{\hb@xt@ #1{\hss#2\hss}}\hskip3pt}
\makeatother
\ctexset{
    % 修改 section。
    section={   
        name={,\quad},
        number={\empty},
        format=\bfseries\centering\zihao{3}, % 设置 section 标题为黑体、右对齐、小4号字
        aftername=\hspace{0pt},
        beforeskip=2ex,
        afterskip=2ex
    },
    % 修改 subsection。
    subsection={   
        name={,\quad},
        number={\arabic{section}.\arabic{subsection}},
        format=\bfseries\zihao{4}, % 设置 subsection 标题为黑体、5号字
        aftername=\hspace{0pt},
        beforeskip=1ex,
        afterskip=2ex
    },
    % 修改 subsubsection。
    subsubsection={   
        name={,\quad},
        number={\arabic{section}.\arabic{subsection}.\arabic{subsubsection}},
        format=\bfseries\zihao{5}, % 设置 subsection 标题为黑体、5号字
        aftername=\hspace{0pt},
        beforeskip=1ex,
        afterskip=1ex
    }
}
\newcommand{\subsubsubsection}[1]{\paragraph{#1}\mbox{}\\}
\setcounter{secnumdepth}{4} % how many sectioning levels to assign numbers to
\setcounter{tocdepth}{4} % how many sectioning levels to show in ToC
% Style definition file generated by highlight 4.8, http://www.andre-simon.de/ 
% highlight theme: Bright
\newcommand{\hldef}[1]{\textcolor[rgb]{0.2,0,0.4}{#1}}
\newcommand{\hlnum}[1]{\textcolor[rgb]{0.2,0.73,0.02}{#1}}
\newcommand{\hlesc}[1]{\textcolor[rgb]{0.65,0.09,0.38}{#1}}
\newcommand{\hlsng}[1]{\textcolor[rgb]{0.09,0.38,0.65}{#1}}
\newcommand{\hlpps}[1]{\textcolor[rgb]{0.4,0.2,0}{#1}}
\newcommand{\hlslc}[1]{\textcolor[rgb]{0,0.4,0.2}{#1}}
\newcommand{\hlcom}[1]{\textcolor[rgb]{0,0.4,0.2}{#1}}
\newcommand{\hlppc}[1]{\textcolor[rgb]{0.33,0.45,0.69}{#1}}
\newcommand{\hlopt}[1]{\textcolor[rgb]{0.33,0.33,0.33}{#1}}
\newcommand{\hlipl}[1]{\textcolor[rgb]{0.65,0.09,0.38}{#1}}
\newcommand{\hllin}[1]{\textcolor[rgb]{0.6,0.6,0.6}{#1}}
\newcommand{\hlerr}[1]{\textcolor[rgb]{1,0,0}{\bf{#1}}}
\newcommand{\hlerm}[1]{\marginpar{\small\itshape\color{red}#1}}
\newcommand{\hlkwa}[1]{\textcolor[rgb]{1,0.19,0.19}{#1}}
\newcommand{\hlkwb}[1]{\textcolor[rgb]{0.96,0.55,0.14}{#1}}
\newcommand{\hlkwc}[1]{\textcolor[rgb]{0,0,1}{#1}}
\newcommand{\hlkwd}[1]{\textcolor[rgb]{0.82,0.11,0.93}{#1}}
\newcommand{\hlkwe}[1]{\textcolor[rgb]{0.87,0.51,0.05}{#1}}
\newcommand{\hlkwf}[1]{\textcolor[rgb]{0.47,0.42,0.72}{#1}}
\definecolor{bgcolor}{rgb}{1,1,1}


\begin{document}

\begin{titlepage}
    \begin{center}

        \zihao{1}\textbf{电子技术实验III\quad 实验报告}\\
        \vspace{0.5cm}
        \zihao{2}\textbf{实验四\quad 模拟乘法器调幅、解调与峰值检波}
    
        \vspace{1.5cm}
        \input{school_badge}
    
        \vspace*{1.35cm}
        \begin{center}
            \hspace{-2em}
            \zihao{4}
            \begin{tabular}{rl}
                \makebox[4em][s]{实验人:}    \hspace{-0.5cm}	&\dlmu[5cm]{王旭东 PB22051030} \vspace{1ex}\\
                \makebox[4em][s]{}    \hspace{-0.5cm}	&\dlmu[5cm]{李\quad 毅 PB22051031} \vspace{1ex}\\
                \makebox[4em][s]{院\quad 系:}    \hspace{-0.5cm}	&\dlmu[5cm]{信息科学技术学院}\vspace{1ex}\\
                \makebox[4em][s]{时\quad 间:}    \hspace{-0.5cm}	&\dlmu[5cm]{2024年11月22日}\vspace{1ex}\\
                \makebox[4em][s]{台\quad 号:}    \hspace{-0.5cm}	&\dlmu[5cm]{26}
                
            \end{tabular}
        \end{center}
    \end{center}
    \end{titlepage}

\newpage
\section{第一部分 \texorpdfstring{\quad}{} 实验目的}
\begin{enumerate}
    \item 了解模拟乘法器的基本工作原理。
    \item 掌握用模拟乘法器(MC1496)实现AM 、 DSB和SSB信号的调制方法。
    \item 掌握模拟乘法器(MC1496)实现AM、 DSB和SSB已调波的解调(同步检波)方法。
    \item 掌握二极管峰值检波电路的实现方法。
\end{enumerate}
    
\section{第二部分 \texorpdfstring{\quad}{} 实验原理}
\begin{enumerate}
    \item \textbf{模拟相乘器芯片---MC1496}

    \begin{figure}[H]
        \centering
        \includegraphics[width=0.7\textwidth]{pics/2.1.png}
        ~\\
        \caption{模拟相乘器芯片---MC1496}\label{fig:2.1}
    \end{figure}

经理论推导,当满足$u_x ≤u_T (26mV)$时, MC1496即可实现两个模拟信号的线性相乘,即
$$U_o=\dfrac{R_C}{R_f U_T} u_x u_y$$

    \item \textbf{模拟乘法器调幅}
    
    设高频载波信号为:$u_c=U_{Cm} cos \omega_c t$;低频调制信号为:$U_\Omega=U_{\Omega m}cos\Omega t$。
    \begin{enumerate}[(1)]
        \item 将$u_\Omega$与一直流$U_{\text{直流}}$叠加后再与$u_c$相乘,则可得到普通调幅信号:
        \begin{equation*}
            \begin{split}
                U_{AM} &=k_1 u_c (U_{\text{直流}}+k_2 u_\Omega) \\
                &= k_1 U_{Cm} cos \omega_c t(U_{\text{直流}}+k_2 U_{\Omega m}cos\Omega t) \\
                &= k_1 U_{\text{直流}} U_{Cm} cos \omega_c t + \dfrac{k_1 k_2}{2}U_{\Omega m}U_{Cm}	[cos(\omega_c+\Omega)t+cos(\omega_c - \Omega)t] \\
            \end{split}
        \end{equation*}
        \item 载波信号$u_c$与调制信号$u_\Omega$直接相乘,可得到抑制载波的DSB信号
        \begin{equation*}
            \begin{split}
                U_{DSB} &=k u_\Omega u_c \\
                &= k U_{\Omega m} U_{Cm} cos\Omega t  cos \omega_c t\\
                &= \dfrac{k}{2}U_{\Omega m}U_{Cm}	[cos(\omega_c+\Omega)t+cos(\omega_c - \Omega)t] \\
            \end{split}
        \end{equation*}
        \item 在DSB信号的输出端再加一级带通滤波器,取出双边带信号的一个边带,则可得到单边带调制SSB信号:
        \begin{equation*}
                U_{SSB}=\dfrac{k}{2}U_{\Omega m}U_{Cm} cos(\omega_c+\Omega)t 
        \end{equation*}
        或
        \begin{equation*}
                U_{SSB}=\dfrac{k}{2}U_{\Omega m}U_{Cm} cos(\omega_c - \Omega)t 
        \end{equation*}
    \end{enumerate}

    \item 模拟乘法器同步检波

    如图,设$u_s=k_3 U_{sm} cos\Omega t cos \omega_c t$,$u_\tau =k_4 U_{\tau m} cos(\omega_\tau t + \phi$。

    当$\omega_c=\omega_\tau$,$\phi=0$时,$u_o=U_{om}cos \Omega t$。

    \begin{figure}[H]
        \centering
        \includegraphics[width=0.45\textwidth]{pics/2.2.png}
        ~\\
        \caption{模拟乘法器同步检波示意图}\label{fig:2.2}
    \end{figure}

    \item 二极管峰值包络检波

    如图为二极管峰值包络检波原理图。利用二极管单向导电特性和RC低通滤波器充放电特性,直接提取出AM波中的包络就还原出调制信号。充电时间常数远小于放电时间常数。

    \begin{figure}[H]
        \centering
        \includegraphics[width=0.6\textwidth]{pics/2.3.png}
        ~\\
        \caption{二极管峰值包络检波原理图}\label{fig:2.3}
    \end{figure}
    
\end{enumerate} 


\section{第三部分 \texorpdfstring{\quad}{} 实验内容及结果}
\textbf{需要说明的是,由于实验时示波器显示的测量数据处于抖动状态,我们的测量方法是按Stop键之后读取数值作为试验记录,之后按Run键再将波形保存为图片,所以试验记录和图片中显示数据可能会有细微差别,报告中的计算全部按照原始数据来计算,图片仅作波形参考}

\subsection{普通调幅(AM)信号的产生与解调}
\subsubsection{AM波形及频谱观测}
调节W1,在TP3得到全载波AM信号如图 \ref{fig:11} 所示,记录信号参数如表 \ref{tab:1} 所示。
\begin{figure}[H]
    \centering
    \includegraphics[width=0.9\textwidth]{pics/11.png}
    \caption{全载波AM信号波形}\label{fig:11}
\end{figure}
\begin{table}[H]
    \centering
    \vspace{-2em}
    \caption{全载波AM信号参数}
    \label{tab:1}
    \begin{tabular}{c|c|c|c|c}
    \hline
               & $Vpp$ & $f$       & $A_{max}$ & $A_{min}$ \\ \hline
    $u_\Omega$ & 500mV & 1.0027KHz &           &           \\ \hline
    $u_c$      & 600mV & 460KHz    &           &           \\ \hline
    $u_{AM}$   &       &           & 1.0125V   & 0.550V    \\ \hline
    $u_o$      & 760mV & 999.4Hz   &           &           \\ \hline
    \end{tabular}
\end{table}

计算得到此时的信号调制度
$$m=\dfrac{A_{max}-A_{min}}{A_{max}+A_{min}}=29.6\%$$

调节W1增大加入$u_\Omega$的直流量,测量数据如表\ref{tab:2}所示,可以观察到,$u_\Omega$直流量越大,AM信号的幅值波动相对于整体均值而言越小,相对应地调制度m越小。
\begin{table}[H]
    \centering
    \caption{不同$u_\Omega$直流量下调制度变化}
    \label{tab:2}
    \begin{tabular}{c|c|c|c}
    \hline
    $u_{\Omega}$直流量 & $A_{max}$ & $A_{min}$ & $m$    \\ \hline
    0.700V          & 1.0125V   & 0.550V    & 29.6\% \\ \hline
    0.764V          & 1.0875V   & 0.625V    & 27.0\% \\ \hline
    0.827V          & 1.1500V   & 0.712V    & 23.5\% \\ \hline
    \end{tabular}
\end{table}

调节同步检波电路移相网络的W5,观测$u_o$波形如图 \ref{fig:14} 所示,记录波形参数为:
$U_{opp}=1.65V,f_o=998.4Hz$。调节W5,观测到$u_o$波形最大不失真时的幅度值为1.67V。
\begin{figure}[H]
    \centering
    \includegraphics[width=0.9\textwidth]{pics/14.png}
    \caption{同步检波后的AM信号波形}\label{fig:14}
\end{figure}

用频谱仪观察AM信号的频谱如图 \ref{fig:15} 所示,可以观察到中心频率处的载波谱峰和两边调制信号的两个谱峰。中心频率与最大边频的频率差值$\Delta f=1KHz$,误差允许范围内,$\Delta f$等于低频调制信号的频率$f_\Omega$。

读取幅度差值 $\Delta A=-16.69dB$,调制度$m=\dfrac{2}{10^{\left\lvert \Delta A\right\rvert/20 }}=29.3\%$,与时域测量得到的调制度$m=29.6\%$在误差允许范围内近似相等。
\begin{figure}[H]
    \centering
    \includegraphics[width=0.85\textwidth]{pics/15.png}
    \caption{AM信号频谱}\label{fig:15}
\end{figure}

改变低频调制信号幅值为300mVpp,观察到AM信号频谱如图 \ref{fig:16} 所示,$\Delta A=-19.85dB$,$m=\dfrac{2}{10^{\left\lvert \Delta A\right\rvert/20 }}=20.3\%$,调制信号幅值减小,调制度减小,与理论分析一致。
\begin{figure}[H]
    \centering
    \includegraphics[width=0.85\textwidth]{pics/16.png}
    \caption{调制信号幅值为300mVpp时的AM信号频谱}\label{fig:16}
\end{figure}
\subsection{抑制载波的双边带(DSB)调幅信号的产生与检波}
用示波器观察记录DSB信号 $u_\Omega,u_C,u_{DSB},u_o$ 的波形如图 \ref{fig:21} 所示,记录信号参数如表 \ref{tab:3} 所示。从图中通道 3 可以较为清晰地看到DSB信号在过零点时的反相现象。
\begin{figure}[H]
    \centering
    \includegraphics[width=0.95\textwidth]{pics/21.png}
    \caption{DSB信号波形}\label{fig:21}
\end{figure}
\begin{table}[H]
    \centering
    \vspace{-2em}
    \caption{DSB信号参数}
    \label{tab:3}
    \begin{tabular}{c|c|c}
    \hline
               & $Vpp$ & $f$       \\ \hline
    $u_\Omega$ & 599mV & 99.985KHz \\ \hline
    $u_C$      & 607mV & 5.020KHz  \\ \hline
    $u_{DSB}$  & 780mV &           \\ \hline
    $u_o$      & 3.0V  & 4.96KHz   \\ \hline
    \end{tabular}
\end{table}

调节同步检波模块的W5,观测$u_o$波形如图 \ref{fig:22} 所示,W5右移,$u_o$幅值增大,$u_o$波形最大不失真时的幅度值为3.76V。

用频谱仪观测DSB信号频谱如图 \ref{fig:23} 所示,可以看出DSB信号具有明显的抑制载波特征,几乎没有中心载波频率处的尖峰,只有两侧调制信号对应的峰。测得DSB信号单侧峰值与中心频率间的差值$\Delta f=4.967KHz$,近似等于调制信号频率$5KHz$
\begin{figure}[H]
    \centering
    \includegraphics[width=0.9\textwidth]{pics/22.png}
    \caption{$u_o$波形}\label{fig:22}
\end{figure}
\begin{figure}[H]
    \centering
    \includegraphics[width=0.9\textwidth]{pics/23.png}
    \caption{DSB信号频谱图}\label{fig:23}
\end{figure}
\subsection{抑制载波的单边带(SSB)调幅信号的产生与检波}
\subsubsection{测试465KHz陶瓷滤波器的幅频特性}
\noindent 用频谱仪观测465KHz陶瓷滤波器的幅频特性曲线如图 \ref{fig:30} 所示,读出滤波器-3dB带宽为10.333KHz。
\begin{figure}[H]
    \centering
    \includegraphics[width=0.7\textwidth]{pics/30.png}
    \caption{465KHz陶瓷滤波器的幅频特性曲线}\label{fig:30}
\end{figure}
\subsubsection{SSB波形及频谱观测}
用示波器观察记录SSB信号 $u_\Omega,u_{DSB},u_{SSB},u_o$ 的波形如图 \ref{fig:31} 所示,记录信号参数如表 \ref{tab:4} 所示。\textbf{此处调节输入信号参数为:$f_c=435KHz,f_\Omega=30KHz$}。
\begin{figure}[H]
    \centering
    \includegraphics[width=0.8\textwidth]{pics/31.png}
    \caption{SSB信号波形}\label{fig:31}
\end{figure}


\begin{table}[H]
    \centering
    \vspace{-1em}
    \caption{SSB信号参数}
    \label{tab:4}
    \begin{tabular}{c|c|c}
    \hline
               & $Vpp$ & $f$       \\ \hline
    $u_\Omega$ & 620mV & 30.135KHz \\ \hline
    $u_{DSB}$  & 700mV &           \\ \hline
    $u_{SSB}$  & 760mV &           \\ \hline
    $u_o$      & 1.33V & 29.996KHz \\ \hline
    \end{tabular}
\end{table}

调节同步检波模块的W5,观测$u_o$波形,W5右移,$u_o$波形整体向右发生相移,未观察到明显幅值变化。$u_o$波形幅度值保持在1.33V左右。

用频谱仪观测SSB信号频谱如图 \ref{fig:33} 所示,可以看出SSB信号具有明显的抑制载波和单边带特征,几乎没有载波频率处的尖峰以及下边带尖峰,只保留了上边带尖峰。测得SSB信号上边带峰值频率为$f=464.67KHz$,与中心频率$435KHz$间的差值$\Delta f=29.67KHz$,近似等于调制信号频率$30KHz$。
\begin{figure}[H]
    \centering
    \includegraphics[width=0.9\textwidth]{pics/33.png}
    \caption{SSB信号频谱图}\label{fig:33}
\end{figure}

\subsection{二极管峰值检波器测试}
观测二极管包络检波$u_{AM}$信号的波形如图 \ref{fig:41} 通道1所示,测量得到:$A_{max}=925mV,A_{min}=487.5mV$,$m=\dfrac{A_{max}-A_{min}}{A_{max}+A_{min}}=30.97\%$,
输出信号$u_o$波形如图 \ref{fig:41} 通道4所示,测量得到输入AM信号正包络变化幅值 $\Delta U_{AMpp}=437.5mV$,输出电压幅值 $U_{opp}=960mV$,交流检波效率$k_\Omega=\dfrac{U_{opp}}{\Delta U_{AMpp}}=2.194$。
\begin{figure}[H]
    \centering
    \includegraphics[width=0.8\textwidth]{pics/41.png}
    \caption{二极管包络检波输入输出信号波形}\label{fig:41}
\end{figure}

\vspace{-1em}
在不同幅值输入高频正弦信号下,用示波器观察记录输入信号波形正半周幅值$U_{cm+}$,用万用表测量检波输出直流电压$V_o$如表 \ref{tab:5} 所示,计算直流检波效率 $k_d$,绘制$k_d\thicksim U_{cm+}$曲线如图 \ref{fig:42} 所示。
\begin{table}[H]
    \centering
    \vspace{-1em}
    \caption{二极管包络检波信号参数}
    \label{tab:5}
    \begin{tabular}{c|c|c|c|c|c}
    \hline
    输入高频幅值(Vpp)  & 1.0    & 1.5    & 2.0    & 2.5    & 3.0    \\ \hline
    $U_{cm+}$(V) & 0.4775 & 0.6875 & 0.9000 & 1.1375 & 1.4400 \\ \hline
    $V_o$(V)     & 0.234  & 0.700  & 1.200  & 1.700  & 2.200  \\ \hline
    $k_d$        & 0.490  & 1.018  & 1.333  & 1.495  & 1.528  \\ \hline
    \end{tabular}
\end{table}
\begin{figure}[H]
    \centering
    \includegraphics[width=0.57\textwidth]{pics/42.png}
    \caption{$k_d\thicksim U_{cm+}$曲线}\label{fig:42}
\end{figure}

用示波器观测检波输出信号的波形,调节二极管包络检波直流负载W3,观察到对角线切割失真现象如图 \ref{fig:51} 通道2所示,记录信号峰峰值为800mV。调节二极管包络检波的交流负载W4,观察到底部切割失真现象如图 \ref{fig:52} 通道2所示,记录信号峰峰值为440mV。
\begin{figure}[H]
    \centering
    \includegraphics[width=0.8\textwidth]{pics/51.png}
    \caption{对角线切割失真现象}\label{fig:51}
\end{figure}
\begin{figure}[H]
    \centering
    \includegraphics[width=0.8\textwidth]{pics/52.png}
    \caption{底部切割失真现象}\label{fig:52}
\end{figure}
\newpage
\section{第四部分 \texorpdfstring{\quad}{} 思考题}
\begin{enumerate}[I.]

    \item \textbf{简述乘法器调幅与集电极调幅/基极调幅有什么异同?}
    
    相同点:都属于幅度调制,通过一个低频信号调节高频信号的幅度。
    
    不同点:
    \begin{enumerate}[(1)]
        \item 乘法器调幅使用乘法器电路,电路较为复杂。集电极调幅/基极调幅通过晶体管的工作特性直接控制信号幅度。电路实现比较简单。
        \item 乘法器调幅使用了乘法器电路,调制的精度较高,灵活性更强,适合对调制精度有较高要求的应用。集电极调幅/基极调幅直接依赖于晶体管的非线性特性,精度相对较低,并且调制深度的控制较为有限。
        \item 乘法器调幅使用乘法器电路,使用差分对,受噪声影响较小,可靠性较高。集电极调幅/基极调幅相对更容易受到噪声的影响,可能导致调制信号失真或幅度变化。
    \end{enumerate}
    
    \item \textbf{简述同步检波与峰值包络检波有什么异同?}
    
    相同点:两种方法均可用于从AM信号中提取原始的调制信号(基带信号)。还原载波信号调制的原始信息。

    不同点:
    \begin{enumerate}[(1)]
    \item 工作原理不同。同步检波通过与载波信号同步的本地振荡器恢复基带信号。峰值包络检波使用二极管和电容提取调制信号的包络。
    
    \item 精度不同。同步检波精度较高,尤其适用于高质量信号传输。峰值包络检波精度较低,可能会受失真和噪声影响。
    
    \item 解调系统复杂度不同。同步检波需要精确的本地载波同步电路,设计复杂。峰值包络检波电路简单,通常只需二极管、电容和电阻即可。
    
    \item 抗噪声能力不同。同步检波抗噪性能较好,适合低信噪比的信号解调。峰值包络检波抗噪性能差,容易受噪声干扰。
    
    \item 对载波同步要求不同。同步检波对载波同步要求高,需要准确的相位和频率匹配。峰值包络检波不需要载波同步,仅检测信号包络。
    
    \item 适用场景不同。同步检波适用于高质量通信,如广播和数字通信系统。峰值包络检波适用于低成本、简单系统,如模拟无线电接收器。
    \end{enumerate}

    

\end{enumerate}

\end{document}